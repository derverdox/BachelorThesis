\chapter{Conclusion}
Blockchain technology was created to solve problems many digital cash protocols failed on.
After years of research and proposals of several Cypherpunks, Satoshi Nakamoto presented a whitepaper about Bitcoin.
The main feature of this solution is a decentralized peer-to-peer network that does not rely on any third party.
Accordingly, it was designed only to process seven transactions per second to focus on decentralization and security in the blockchain trilemma.
Every attempt to change the blockchain design by other projects like Ethereum to achieve more scalability leads to less security or centralization.
However, both are needed to make the chain a tamper-proof digital cash solution, which it was designed for.
Further, the social consensus of the Bitcoin network is the backbone of its value proposition.
The decentralized open-source character, the origin story, and its network effects differentiate it from every other cryptocurrency implemented afterward.
If other cryptocurrency and blockchain implementations do not contribute any improvements, they might just be trying to imitate Bitcoin and its narrative for marketing purposes.

As a closed ecosystem, the blockchain only contains data secured by its consensus rules.
Accordingly, it cannot prove the integrity of external data in any way.
This so-called blockchain oracle problem leads to data only being valid when an oracle grants its validity.
Ingame items on the blockchain are only valuable if the game's code uses them. 
If an entity does not abide by the terms of a smart contract, it always needs an external authority to enforce it.
In addition, there is no guarantee that a smart contract that is created today will be used for its initial purpose tomorrow.
Consequently, third parties are needed again, even though the entire technology should exclude them, and therefore, they must accept significant disadvantages to achieve that.
These drawbacks also come into play when using smart contracts instead of a centralized database to store game-related data. 
Although there are some benefits, it is questionable whether they are significant enough to justify the software engineering problems that come with them.

As a result, we conclude that blockchain technology is a superb implementation for solving the problems of digital cash.
However, we do not see any use case for blockchains as a distributed database to store external data.
The only data that should be stored on a blockchain is data that was created through a decentralized consensus that is very hard to compromise.
Proof of Stake removes the thermodynamical properties provided by Proof of Work, making it impossible to identify the honest chain by its mining difficulty.
Therefore, blockchain networks, namely Ethereum, that use the Proof of Stake consensus algorithm cannot store digital wealth.
While it is primarily used to execute smart contracts, which are subject to the Oracle problem, it is even questionable whether Ethereum was designed to solve an existing problem or if the need for smart contracts in a blockchain was constructed to justify Ethereum's existence.