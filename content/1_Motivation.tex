\chapter{Motivation}

Computers and the Internet play an essential role in today's world. 
Many things in modern society would only work with digital technology.
Even back in 2004, Hoffman, Novak, and Venkatesh raised the question of whether the Internet has become indispensable, meaning the integration into one's daily life \cite{hoffman2004}.
Nearly two decades later, this question is still relevant.
With a modern computer or mobile phone, one can access publicly available information quickly because every bit of information is copyable on a digital device.
Viewing a website, sending an email, or creating a post on X is only possible by copying data.
However, the fact that every bit is duplicatable creates a new untrivial problem of digital information.
Financial systems rely on the immutability of completed transactions. 
No one would buy a stock if it were just a file that everyone could easily copy.
Spending cash online would not make sense if anyone could duplicate a dollar bill before using it for an online transaction.
The standard way to solve this problem is to establish a trusted third party that would monitor everything and log the rightful owner of the file.
Still, new problems arise when using a system based on trust.

In 2008, Satoshi Nakamoto published a whitepaper to a mailing list of cryptographers on metzdowd.com \cite{nakamotoemail_1}.
The paper is called "Bitcoin: A Peer-to-Peer Electronic Cash System" \cite{nakamoto2008} and proposes a solution to this problem. 
It includes the concept of a digital payment system that uses a so-called blockchain.
This new technology should solve issues of digital cash like double-spending \cite{chowan2021}, privacy, and the need for trusted third parties \cite{Douceur2002}.
The blockchain would not only be used to run the Bitcoin protocol but was also the discovery of digital scarcity.
A bitcoin is not copyable because of the immutability of the transaction record made possible by the blockchain \cite{popovski2014}.
Although, are there other fields of application for this solution?

More than a decade later, many industries are trying to use the blockchain to enhance existing systems, one being the gaming industry \cite{fowler2021}.
As a game developer, the evaluation of the practicality and the impact of blockchain technology in game development raised my interest.