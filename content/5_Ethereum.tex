\chapter{Ethereum and Smart Contracts}
The Ethereum blockchain allows code execution designed in a smart contract language like \textbf{Solidity}. 
As a result, a game can execute primarily defined methods of the smart contract. 
Using these contracts makes it possible to store in-game items, weapons, or character statistics on Ethereum. \cite{buterin2014}

\section{Ethereum Virtual Machine}
Like Bitcoin, Ethereum is a decentralized network that agrees on a network state.
A state is a collection of accounts with a 20-byte address.
Every account stores its Ether balance, a contract code if one was provided in bytecode form, and a contract storage, which is divided into three types.
First, there is long-term storage that uses key-value pairs to store data.
To create variables temporarily, developers can use the stack or the memory.

Further, two types of accounts are described. The first one is a standard user account controlled with a private key. 
They only issue external transactions, meaning transactions from outside the blockchain.
The second one is a contract account that controls its smart contract. 
They can also issue internal transactions from one smart contract to another.
Receiving a transaction as a contract account always leads to a code execution and, thus, to a state change. \cite{ethereum_virtual_machine}
The fact that the blockchain executes Turing-complete code leads to the term \textbf{Ethereum Virtual Machine} because the network acts as one computer system.

However, code execution does not happen instantaneously. 
Like regular transactions, a code execution transaction needs verification in a block.
Thus, standard money transfers and code executions compete for block space, resulting in transaction fees, called gas, measured in wei. 
One wei equals 0.000000000000000001 Ether.
Depending on the network usage, the cost of gas changes.
In addition, a user can provide a gas limit when issuing a transaction to prevent unintended execution of poorly programmed smart contracts.
As a result, developers of smart contracts are heavily incentivized to write high-performance code that uses storage effectively.
Consequently, infinite loops are avoided. 
Crashing the whole network would cost a tremendous amount of Ether, which makes it a wasteful effort.

\section{Smart Contracts in Game Development}
Since the EVM bytecode is Turing-complete, many different things can be done using programming languages like Solidity.
This chapter will show some approaches to implementing game-related data using Solidity, the most commonly used smart contract language.
It is a high-level, object-oriented, statically typed programming language.
It supports various concepts of object-oriented programming, such as inheritance, interfaces, abstract classes, and structures.

\clearpage
\subsection{Smart Contract Implementation: Virtual Pets}

\begin{framed}
\textbf{Implementation: } Virtual Pet \medskip \hrule \medskip
\textbf{Used Contract: } ERC-721 Non-Fungible Token Standard\medskip \hrule \medskip
\textbf{Details:}  This example shows how to implement so-called Virtual Pets using Solidity.
The contract is based on the ERC-721 NFT standard. 
Everyone can create a new pet by using the createNewPet() function.
The function mints a new ERC-721 token and randomly draws properties for the new pet.
\end{framed}
First, some pet properties are defined in the contract.
\lstinputlisting[language=Solidity,linerange={3-16}, caption=Solidity pet properties, label=Solidity pet properties]{listings/VirtualPet.sol}

However, every pet should have different properties, so functions to generate them randomly are needed.
In addition, pet sizes should have different probabilities.
\lstinputlisting[language=Solidity,linerange={43-54}, caption=Solidty pet property creation, label=Solidty pet property creation]{listings/VirtualPet.sol}
\lstinputlisting[language=Solidity,linerange={56-67}, caption=Solidty pet property random function, label=Solidty pet property random function]{listings/VirtualPet.sol}

The random number functions used for this example are deterministic due to the nature of the EVM not having any byte code functionality to create true randomness.
\lstinputlisting[language=Solidity,linerange={71-78}, caption=Solidity random functions, label=Solidity random functions]{listings/VirtualPet.sol}

Finally, a function to create a new pet is implemented. 
In addition, a function to obtain the pet properties is also needed so they can be used in a game.
\lstinputlisting[language=Solidity,linerange={18-35}, caption=Solidty pet creation function, label=Solidty pet creation function]{listings/VirtualPet.sol}
\newpage
Then, a getter for external function calls from the Unity game implementation is created.
\lstinputlisting[language=Solidity,linerange={36-39}, caption=Solidity pet property getter, label=Solidity pet property getter]{listings/VirtualPet.sol}
\clearpage

\subsection{Unity Implementation: Using Virtual Pets in a Game}
After publishing the virtual pet smart contract on an EVM blockchain like Ethereum, a game engine like Unity can communicate with it.
For this purpose, the Thirdweb SDK will be used \cite{thirdweb_sdk}.

Code to set up the player wallet for further smart contract interactions.
\lstinputlisting[language={[Sharp]C},linerange={3-20},caption=Setup Code, label=Setup Code]{listings/UnityImplementation.cs}

Example code to call a read function on the smart contract and do something afterward.
\lstinputlisting[language={[Sharp]C},linerange={22-29},caption=Read interaction, label=Read interaction]{listings/UnityImplementation.cs}

\newpage
Example code to call a write function on the smart contract and do something afterward.
\lstinputlisting[language={[Sharp]C},linerange={31-39},caption=Write Interaction, label=Write Interaction]{listings/UnityImplementation.cs}

Helper function to obtain an object representation of the smart contract.
\lstinputlisting[language={[Sharp]C},linerange={43-46},caption=Smart contract loading, label=Smart contract loading]{listings/UnityImplementation.cs}

Helper function to call the getPetProperties() function of the smart contract.
\lstinputlisting[language={[Sharp]C},linerange={48-51},caption=Get per properties, label=Get per properties]{listings/UnityImplementation.cs}

Helper function to call the createNewPet() function of the smart contract.
\lstinputlisting[language={[Sharp]C},linerange={53-56},caption=Call create new pet function, label=Call create new pet function]{listings/UnityImplementation.cs}