\chapter{Future Work}

This thesis might serve as a foundation to further expand on what blockchains can be used for.
While storing external data on a blockchain does not guarantee the truthfulness of the information, there might still be other use cases for blockchains.
We propose a different view to concentrate on second-layer solutions that do not change a blockchain or store any third-party information directly.
Instead, layer two protocols could use the blockchain as a distributed timestamping server, which it was initially called by Nakamoto \cite{nakamoto2008}, or as a base layer to prove the state of their network.
These layers could also choose a more scalable design since the underlying chain is still secure.
Storing third-party data on these new systems has no drawbacks like storing them on the main chain, making it a better software engineering approach.

\section{Scaling Bitcoin with the Lightning Network}
The Bitcoin Blockchain creates a fundament for a whole new ecosystem to be built on top. 
Since its first launch over a decade ago, new systems have been created with Bitcoin as a base settlement layer.
One of them is the Bitcoin Lightning network \cite{poon2016}, which is capable of routing payments through a network of payment channels that are settled on the Bitcoin blockchain.
Every payment channel is a wallet that is controlled by two entities. 
They can transfer coins inside this channel without waiting for the Bitcoin blockchain to verify it in a new block. 
This results in billions of daily transactions compared to about seven transactions per second on the blockchain. 
Thus, games could use Bitcoin Lightning as an alternative to a traditional payment infrastructure for micro-payments.

\section{Trusted Timestamping}
Timestamping is used to prove data's existence at a particular moment.
Thus, the services that provide such timestamps must be trustworthy.
In addition, the timestamps must be cryptographically saved from being altered afterward.
That is where Bitcoin could play an essential role as a distributed time-stamping server.
The idea of using the Bitcoin blockchain like that is further discussed in \cite{timestamping}.
The author proposes to store hashes of documents or data on the Bitcoin blockchain, which proves the existence of the data at the time of the block validation.
As already discussed, storing data on the main chain comes with drawbacks.
However, the only property a blockchain can prove about third-party data is when it was stored.
If this is the only property needed and the transaction costs are worth the data, it might be a valid use case that needs further research.

\section{BitVM}
Instead of storing data directly on a blockchain, BitVM could be used as a second layer that replaces implementations like Ethereum.
We have already concluded that smart contracts on the Ethereum platform are subject to its scalability issues.
Further, the contracts need oracles in most cases, creating a contradiction between the initial idea of excluding third parties and the need for trusted data providers and contract enforcers.
However, when a second layer is used instead, the software engineering drawbacks are not a problem anymore.
While the use case of smart contracts is still questionable, the blockchain is no longer used to execute them.
Second-layer smart contract platforms might serve new application areas that need to be discovered by further research in this field.