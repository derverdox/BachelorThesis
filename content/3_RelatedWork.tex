\chapter{Related Work}

\section{The Bitcoin Blockchain}
None of the technologies and ideas could solve all the problems of digital cash simultaneously. 
Yet, the need for such a system grew even further due to the financial crisis starting in 2007.
The crisis had several causes. 
Some of them were a housing price bubble in the US caused by the banking sector, a flawed regulatory environment, and the downfall of Lehman Brothers. \cite{baily2009origins}
Consequently, the Cypherpunks again saw the flaws of a centralized credit-based money system.
This led Satoshi Nakamoto, an anonymous programmer, to devise an implementation to solve the problems that made former concepts
unsuitable as digital money.
As stated in the Bitcoin talk forum, Nakamoto's ideas were greatly influenced by former concepts.
\begin{quote}
    \centering 
    \textit{Bitcoin is an implementation of Wei Dai's b-money proposal […] on Cypherpunks […] in 1998 and Nick Szabo's Bitgold proposal[…].} \cite{satoshi_quote_1}
\end{quote}

Although the Bitcoin whitepaper was released on the thirty-first of October in 2008, the software and its code were made open source on the third of January in 2009.
It is the same day Satoshi Nakamoto launched the program for the first time and mined the first Bitcoin block.
This so-called genesis block also contained a quote referencing the crisis.

\begin{center}
    \begin{quote}
        \centering 
        \textit{Chancellor on brink of second bailout for banks.} \cite{editor_2009_chancellor}
    \end{quote}
\end{center}

\section{Ethereum}
In 2014, Vitalik Buterin, a programmer and co-founder of Bitcoin Magazine, published a whitepaper about a protocol he named Ethereum.
In his whitepaper, the author first points out different projects that use the decentralized design of Bitcoin to create new projects. 
He concludes that to build a new decentralized protocol, one could either build a new network with new consensus rules or a second layer on top of Bitcoin.
While Bitcoin has a scripting language, it is not Turing-complete. 
Essential concepts like loops do not exist to prevent infinite loops from crashing network nodes. 
While this design choice makes Bitcoin resilient against attacks, it also limits the possibilities of second-layer solutions.
Consequently, he envisioned a new blockchain-based system to execute Turing-complete byte code. 
Thus, developers could implement so-called smart contracts, first described by Nick Szabo \cite{szabo1994}, that are stored and run on the blockchain. \cite{buterin2014}
Szabo described a smart contract as a digital representation or implementation of legal contracts formalized in code.
The potential of so-called non-fungible tokens in games implemented using smart contracts is further discussed in \cite{fowler2021}.

Ethereum would be the base layer for so-called decentralized applications (dapps).
Over 42 days in 2014, Ethereum was funded via a presale. One could buy 2000ETH for around 500 us dollars.
However, the price was raised step by step later.
Sixty million Ether were created for the presale, and only 80 percent were sold to the public, while the Ethereum Foundation kept the rest. \cite{eth_presale}

Although the Bitcoin blockchain is only capable of executing basic operations, there are attempts to change this.
Robin Linus released a paper about BitVM in October of 2023. 
While Ethereum executes the code on its network nodes, BitVM only uses the Bitcoin Blockchain to verify them.
As a result, Bitcoin could entirely replace Ethereum's smart contracts, thus making them obsolete. \cite{linus2023}